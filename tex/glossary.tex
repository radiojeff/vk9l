% Glossary of terms that are used in the scope document

\longnewglossaryentry{vna}{ name ={{VNA}}}
{
A Vector Network Analyzer (VNA) is a device that measures the power and characteristics of a high-speed signal, such as amplitude and phase, going into and out of a component or network. It can be used to measure the impedance of an antenna, which allows you to design a matching circuit to optimize the antenna's electrical match to its feed. 
}

\longnewglossaryentry{qrv}{ name={{QRV}} }
{
To begin operation with at least one station on the air
actively taking CQ requests and logging.
}


\longnewglossaryentry{hf} { name={{HF}} }
{
High frequency for amateur bands ranges that apply to the expedition
are between 160 meters through 6 meters.  The Australian amateur
radio license will not permit operation on 60 meters.}

\longnewglossaryentry{iota} { name={{IOTA}} }
{Islands on the Air. 
Established in 1964, it promotes radio contacts with stations located on islands around the world to enrich the experience of all active on the amateur bands and, to do this, it draws on the widespread mystique surrounding islands.}


\longnewglossaryentry{rsgb} { name={{RSGB}} }
{
The Radio Society of Great Britain (RSGB) is the United Kingdom's recognized national society for amateur radio operators. The society was founded in 1913 as the London Wireless Club, making it one of the oldest organizations of its kind in the world.
}


\longnewglossaryentry{lhi} { name={{LHI}} }
{
Lord Howe Island.
}


\longnewglossaryentry{crankir} { name={{CrankIR}} }
{
The Crank-IR is a vertical antenna made by SteppIR.
\par
The SteppIR CrankIR is a portable Ham radio antenna that can be used for park activations. It has a standard model that can be used for 6M through 40M, and it also has an optional tunable Radial Unit. The antenna is lightweight, high performance, and extremely portable. It has a patented folded design that allows for a 40\% reduction in size with only a 0.3dB reduction in performance compared to a full length vertical.
}

\longnewglossaryentry{multisingle} { name={{M/S}} }
{
Multi-single (M/S) refers to an operational condition where multiple
operators are using a single radio.
}


\longnewglossaryentry{singletwo} { name={{SO2R}} }
{
Single Operator, Two Radios.  In this scenario a single operator
uses two radios, simultaneously.
}

\longnewglossaryentry{flyin} { name={{Fly-In}} }
{
A reference to a DX'pedition where the team can use commercial
or chartered aircraft to land on the site for operation.  No marine
vessels are required.  A beach-landing on an island by Zodiac or other
water-craft is NOT fly-in.   A helicopter lift operation from a deck
of a sea trawler is NOT fly-in.   
\par
Fly-in DX'peditions are noted to be a bit easier because except for
terrestrial travel from the airport to the operation site the bulk of the
travel was by commercial or chartered aircraft.
}

\longnewglossaryentry{beam} { name={{Beam}} }
{ 
An antenna configuration where the main criteria is that the front to back
gain ratio is pronounced.  A vertical typically does not have this 
condition unless it is part of an array.  See VDA.
}

\longnewglossaryentry{lotw} { name={{LoTW}} }
{
Log Book of the World.  A free service provided by the ARRL
to let amateur radio operators log contacts and receive
confirmation for the QSO.  Contacts collected and verified
in the LoTW system are automatically eligible for credit
for seeking ARRL contest awards, such as DXCC.
}

\longnewglossaryentry{qsl} { name={{QSL}} }
{
Confirmation of the QSO either via electronic means
or more classically by paper cards with the pertinent
information that is required by the officially sanctioned
card-checker system.  For example, ARRL DXCC card checkers
are volunteers who check the authenticity of the QSL records
and help file the paperwork with the ARRL to grant credit
for those QSO as applicable.  QSL'ing is itself an important
part of the DX'pedition.  Amateurs want confirmation either
in LoTW form or QSL card.}

\longnewglossaryentry{eme} { name={{EME}} }
{
Moon-bounce contact.  Otherwise known as Earth-Moon-Earth.  A
highly coordinated method of sending and receiving messages
over usually higher frequencies (like 6 meters) where the moon
itself reflects the signal back to earth.
}

\longnewglossaryentry{wsjtx} { name={{WSJT-X}} }
{WSJT-X is a software package that is used to code and decode 
digital modes in amateur radio.  It is a popular software program
that has facilitated FT-8 mode throughout amateur radio use for the 
past several years.   A common add-on to WSJT-X is JT-Alert.
\par
For most DX operations, the DX station will run in F/H (Fox Hound)
mode where the DX operation is the Fox and the general public
amateur stations represent the Hound.   The software WSJT-X has a 
dedicated setting to enable F/H mode.   For regular modes
as well as F/H mode -- the software with some configuration required --
will code (send) and decode (receive) the data and render it for the
user -- both the DX station and the DX chasing station.
\par
The ARRL QST magazine has published several excellent articles about
FT-8 and WSJT-X.
\par
In particular the article by Al Rovner, K7AR --
OCT 2023 - QST (PG. 53),
An Introduction to WSJT's DXpedition Mode. 
\par
There is a lot to unpack but the software is very enjoyable to use.
}

\longnewglossaryentry{dqrm} { name={{DQRM}} }
{
A small number of stations generate Deliberate QRM, known as DQRM, by transmitting on the frequency of a rare station in order to disrupt the operation. They do so anonymously, not identifying with their licensed call-sign and thereby contravening the terms of their transmitting licence.
}

\longnewglossaryentry{atno} { name={{ATNO}} }
{All Time New One.  The first time an amateur logged
a DX entity, that is what is known as an ATNO.  For example,
if an amateur ``worked'' (contacted and exchanged information)
with TX5S and never before had worked Clipperton Island, then TX5S
would be a ATNO for Clipperton Island.}

\longnewglossaryentry{oqrs} { name={{OQRS}} }
{ An online system provided by third parties to help
the amateur secure an authenticated QSL from the DX station.
It either refers to a request for a QSL card or request that the QSO
log be uploaded to an authorized body that can likewise provide
authentication of the QSO for contest purposes.  An example
is H44WA -- using the OQRS system at ClubLub, an amateur can
request a card and/or request the log be uploaded to LotW.  
\par
Usually OQRS expect a small donation.  In the case of a paper QSL
card, a donation is often required.}

\longnewglossaryentry{tene} { name={{Ten Essentials}} }
{
In backpacking the \href{https://www.mountaineers.org/blog/what-are-the-ten-essentials}{Ten Essentials} are below:

\par
\begin{enumerate}
\item {\textbf{Navigation:}} map, altimeter, compass, (GPS device),
 (PLB, satellite communicator, or satellite phone), (extra batteries or battery pack)
\item {\textbf{Headlamp:}} plus extra batteries
\item {\textbf{Sun protection:}} sunglasses, sun-protective clothes, and sunscreen
\item {\textbf{First aid:}} including foot care and insect repellent (if required)
\item {\textbf{Knife:}} plus repair kit
\item {\textbf{Fire:}} matches, lighter and tinder, or stove as appropriate
\item {\textbf{Shelter:}} carried at all times (can be a lightweight emergency bivy)
\item {\textbf{Extra food:}} beyond minimum expectation
\item {\textbf{Extra water:}} beyond minimum expectation, or the means to purify
\item {\textbf{Extra clothes:}} sufficient to survive an emergency overnight
\end{enumerate}

\par
As this applies to amateur radio DX'peditions is subtle.
\par
\begin{enumerate}
\item {\textbf{Navigation:}} Depending on where the intended location is
for the operation, all of the above could apply.   When the location
is to a resort or lodge or location that has heavy traffic then some of
the items are obviously not required.   Then again, in emergency -- if 
one had to relocate then even if the original location is well known, if the
demands for relocation put the DX'pedition in a strange new location, those
items would be useful.   At a minimum -- a map and a compass.  In the modern
era, a GPS might serve also.
\item All of these apply to DX'peditions no matter
where or when the operation occurs.  Exactly as noted above. Do not even debate it.
{\textbf{Headlamp,}}
{\textbf{Sun protection,}}
{\textbf{First aid,}} 
{\textbf{Extra food,}} 
{\textbf{Extra water,}} 
{\textbf{Extra clothes,}} 
\item {\textbf{Knife:}} plus repair kit.  It is difficult to carry-on this
item (i.e., impossible) but the item should be part of your checked-bag if
that is part of the travel arrangement.
\item {\textbf{Fire:}} Even in a checked-bag situation, this can be
a dangerous thing to bring.  So re-calibrate this -- in checked-bag
a butane lighter might be passable.  If the expedition is truly camp-style
or backpacking style (tent and generator), then all of the original backpacking
list items are required.
\item {\textbf{Shelter:}} This can be as simple as a reflective heat trapping
sheet.  In emergency it can serve as a shield from the sun, a barrier from
the rain, and a signal marker for possible rescue.  It can save your life.
\end{enumerate}
\par
Whatever the case may be, no matter where you go:  Be Prepared.
\par
This expedition to VK9L is going to bring those items that are deemed
worth while for safety but not excessive given the amount of
support that is available on the island.
\par
In other words -- put snack bars in your carry on luggage.  You may be 
stuck at the airport waiting for a delayed flight and no food is available
in your terminal for hours.
}

\longnewglossaryentry{vda} { name={{VDA}} }
{
Vertical Dipole Array.  In the simplest form, the deployment of
a two element vertical array.  One element is the D.E. - {\textit{driven element}}
and the other element is the P.E. {\textit{parasitic element}}.
The  D.E. is fed with coaxial cable to a balun then to the halves of the
D.E.  The P.E. is a single conducting wire or tube.  The two vertical
elements D.E. and P.E. are spaced apart and guyed.   Careful tuning
and adjustments made produce a very effective antenna system for 
deployments near salt-water.  Gain in the neighborhood of 9:1 is not
unheard of.  
\par
The French design variant is similar.  Instead of two conducting vertical
elements, there is a single non-conductive mast and the D.E. and P.E. are wire suspended from the apex of the mast to the base of the mast, spread apart mid-way down with a horizontal boom (also non-conductive).  The shape of the wire
(conducting element material) is rhombic.    The gain is not as high
as the metal version, but the speed at which the French VDA design can be
raised and the simplicity of the design make it a preferred choice for
a lot of DX'peditions.
\par
Both styles are only effective near salt-water.  Away from salt-water
they do not have nearly the same gain potential.  They are not 
recommended for locations that are near buildings, or too far from salt
water.  Within a couple hundred feet of salt-water is probably close enough.
The beneficial effect is much better the closer the salt water is located.
}


\longnewglossaryentry{oth} { name={{OTH}} }
{
Over the Horizon.  Usually refers to radar signal from
sovereign states or military operations.  It's a wide band, loud,
and terrible noise.  In most cases, the OTH Radar signal is short time span.
}
